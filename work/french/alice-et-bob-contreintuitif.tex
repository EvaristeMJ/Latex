\documentclass[a4paper]{article}
\usepackage{amsmath, amssymb, amsthm}
\usepackage{mdframed}
\usepackage{tikz}
\usepackage{tikz-cd}
\usepackage{hyperref}

\usepackage{mdframed}

\newcommand*{\R}{\mathbb{R}}
\newcommand*{\C}{\mathbb{C}}
\newcommand*{\N}{\mathbb{N}}
\newcommand*{\Z}{\mathbb{Z}}
\newcommand*{\Q}{\mathbb{Q}}
\newcommand*{\F}{\mathbb{F}}
\newcommand*{\P}{\mathbb{P}}
\newcommand*{\K}{\mathbb{K}}

\begin{document}
Supposons que Alice et Bob jouent à un jeu. Ils réalisent \(100\) lancers de pièces.
Alice lit les résultats de lancer des pièces \(1,2,...,100\) et Bob lit les résultats 
de lancer des pièces \(1,3,5,...,99\) puis des pièces \(2,4,6,...,100\). Ils lisent 
simultanément les pièces (donc Alice lit 1, Bob lit 1, Alice lit 2, Bob lit 3, etc). Dès 
que l'un d'eux lit deux piles de suite, il gagne. Qui a le plus de chance de gagner?\\

Tant que Bob perd, Alice lit une séquence de résultats conditionnées. Pour résumer, Alice
deux cas :
Si c'est une pièce paire, elle est entre deux pièces déjà lues par Bob. Si on note 
\(X\) cette pièce, elle observera \(FXP\), \(FXF\) ou \(PXF\). Remarquons qu'elle ne peut pas gagner avec \(FXF\).
\begin{equation}
    \P(A_i) = \P(B_i^c)\P(A_i|B^c_i) = \frac{1}{4}\P(B_i^c).
\end{equation}

\end{document}