\documentclass{article}[a4paper]
\usepackage{biblatex}
\usepackage{svg}
\usepackage{graphicx} % Required for inserting images
\usepackage{amsmath} 
\usepackage{adjustbox}
\usepackage{amssymb}
\usepackage{tikz,tkz-tab,tikz-cd} 
\usepackage{mathrsfs}
\usepackage{amsthm}
\newcommand{\R}{\mathbb{R}}
\newcommand{\C}{\mathbb{C}}
\newcommand{\N}{\mathbb{N}}
\newcommand{\Z}{\mathbb{Z}}
\newcommand{\Q}{\mathbb{Q}}
\newcommand{\K}{\mathbb{K}}
\newcommand{\norm}[1]{\lVert#1\rVert}
\newcommand{\braket}[2]{\langle #1,#2\rangle}
\renewcommand{\P}{\mathbb{P}}
\newcommand{\E}{\mathbb{E}}
\title{Représentation et marche aléatoire}
\begin{document}
\begin{abstract}
    Ce sujet traite de la théorie des représentations pour des groupes abéliens finis. 
    Une application proposée d'une telle théorie est la définition d'une transformée 
    de Fourier sur des groupes généralisant de nombreux résultats que vous avez pu 
    observer en physique. C'est l'occasion d'introduire les bases de l'analyse harmonique.
\end{abstract}
Dans tout le sujet, \((G,+)\) désignera un groupe abélien de cardinal \(n\). 
On définit également l'ensemble des morphismes de groupe \(\widehat{G}\) de \((G,+)\) 
vers \((\mathbb{C}^\times,\times)\) que l'on appelera \textit{groupe dual}.
\\
L'étude portera aussi sur le \(\C\)-espace vectoriel \(E = \C^G\) que l'on appelera 
espace dual. Sur cet espace, nous définissons deux applications. 
La première, notée \(\braket{.}{.} : E\times E \to \C\), sera appelée produit hermitien 
et sera définie par 
\begin{equation*}
    \braket{f}{g} = \frac{1}{n}\sum_{x\in G} f(x)\overline{g(x)}.
\end{equation*}
La seconde, notée \(\cdot\star \cdot : E\times E \to E\), sera appelée produit de 
convolution et sera définie par 
\begin{equation*}
    \forall x\in G, (f\star g)(x) = \frac{1}{n}\sum_{y\in G} f(y)g(x-y).
\end{equation*}
\section{Préliminaires}
L'objectif de cette partie est de démontrer quelques propriétés essentielles.
Un produit hermitien est une généralisation du produit scalaire appliqué à des espaces 
complexes. Soit \(F\) un espace vectoriel complexe, une application 
\(\phi : F\times F \to \C\) est appelée produit hermitien si elle vérifie les propriétés suivantes 
pour tout \(f,g,h\in F\) et pour tout \(\lambda\in \C\):
\begin{itemize}
    \item \(\phi(f,g) = \overline{\braket{g}{f}}\),
    \item \(\phi(f+\lambda g,h) = \phi(f,h)+\lambda\phi(g,h)\),
    \item \(\phi(f,f) \in \R_+\),
    \item \(\phi(f,f) = 0\) si et seulement si \(f = 0\).
\end{itemize}
On montre quelques résultats sur ces produits.
\begin{enumerate}
    \item Soit \(\phi\) un produit hermitien sur \(F\), montrer que pour tout \(f,g,h\in F\) et pour tout \(\lambda\in \C\),
    \begin{equation*}
        \phi(f,g+ \lambda h) = \phi(f,g) + \overline{\lambda}\phi(f,h)
    \end{equation*}
    \item On dit qu'une famille \(f_1,...,f_k\) est orthonormale si pour tout \(i,j\in \{1,...,k\},
    \braket{f_i}{f_j} = \delta_{i,j}\). Montrer qu'une famille orthonormale est libre.
    \item Montrer que \(\braket{\cdot}{\cdot}\) est un produit hermitien sur \(E\).
    

    On admet que \(\varphi\) sur \(\C^k\times \C^k\) défini par 
    \begin{equation*}
        \forall x,y\in \C^k, \varphi{x}{y} = \sum_{i=1}^k x_i\overline{y_i}
    \end{equation*}
    est un produit hermitien.
    \item Une matrice \(U\in \mathcal{M}_k(\C)\) est dite unitaire si les colonnes de \(U\) forment 
    une base orthonormale de \(\C^k\). 
    Montrer qu'une matrice \(U\in \mathcal{M}_k(\C)\) est unitaire
    si et seulement si elle vérifie \(\overline{U^T}U = I_k\).
    \item Montrer que les vecteurs propres normalisés d'une matrice unitaire
    forment une famille orthonormale de \(\C^k\).

\end{enumerate}
\subsection*{Correction de la première partie}
\begin{enumerate}
    \item On applique les différents axiomes
    \begin{align*}
        \phi(f,g+\lambda h) &= \overline{\phi(g+\lambda h,f)}\\
        &= \overline{\phi(g,f)+\lambda\phi(h,f)}\\
        &= \overline{\phi(g,f)}+\overline{\lambda}\overline{\phi(h,f)}\\
        &= \phi(f,g) + \overline{\lambda}\phi(f,h).
    \end{align*}
    C'est ce qu'on appelle aussi la sesquilinéarité droite.
    \item On suppose que \(\sum_{i=1}^k \lambda_i f_i = 0\) où \(\lambda_i\in \C\).
    On a 
    \begin{align*}
        0 &= \braket{\sum_{i=1}^k \lambda_i f_i}{f_j}\\
        &= \sum_{i=1}^k \lambda_i \braket{f_i}{f_j}\\
        &= \lambda_j.
    \end{align*}
    \item On vérifie les propriétés.
    \item Il suffit de faire les calculs. 
    \begin{align*}
        (\overline{U^T}U)_{i,j} & = \sum_{l=1}^k \overline{U_{l,i}}U_{l,j}\\
        & = \braket{U_j}{U_i}\\
        & = \delta_{i,j}.
    \end{align*}
    On a donc bien \(\overline{U^T}U = I_k\). La réciproque est immédiate.
    \item Soit \(\lambda\) une valeur propre de \(U\) et \(v\) un vecteur propre normalisé associé. \(\braket{Ux}{Uy} = \overline{(Uy)^T}Ux = \overline{y^T}\overline{U^T}Ux = \braket{x}{y}\).
    En particulier, \(\braket{Uv}{Uv} = \braket{v}{v} = \lvert\lambda\rvert^2 \braket{v}{v}\). Donc \(\lvert\lambda\rvert^2 = 1\).
    \item Soit \(\lambda,\mu\) deux valeurs propres distinctes de \(U\) associées à \(v,w\).
    On a 
    \begin{align*}
        \braket{v}{w} &= \braket{Uv}{Uw}\\
        &= \lambda\overline{\mu}\braket{v}{w}.
    \end{align*}
    Si \(\braket{v}{w} \neq 0\), alors \(\lambda = \overline{\mu}^{-1}\). 
    Or, \(\lvert\lambda\rvert = 1\), donc \(\mu = \overline{\overline{\mu}} = \lambda\).
\end{enumerate}
\section{Convolutions de fonctions}
\begin{enumerate}
    \item Montrer que \(\star\) est une loi commutative, associative et bilinéaire sur \(E\).
    \item 
\end{enumerate}
\section{Caractères et groupe dual}
L'objectif de cette partie est de déterminer une base orthonormale de \(E\) pour le produit hermitien.
Cela se fera via l'étude de l'endormorphisme de translation \(\tau_y : E\to E\) avec \(y\in G\) défini par
\begin{equation*}
    \forall x\in G, \tau_y(f)(x) = f(x-y).
\end{equation*}
\begin{enumerate}
    \item Montrer que \(y\mapsto \tau_y\) est un morphisme de groupe de \(G\) dans \(\mathcal{GL}(E)\).
    \item Montrer que \(T_y\) est unitaire pour tout \(y\in G\).
    \item Montrer que si \(\chi \in \widehat{G}\), i.e. \(\chi\) est un morphisme de groupe de \(G\) dans \(\C^\times\), 
    alors \(\chi\) est un vecteur propre de \(\tau_y\).
    \item Montrer que \(\tau_y\) est diagonalisable.
    \item En déduire qu'il existe \(\chi_1,\dots,\chi_n\in \widehat{G}\) distincts tels que 
    \begin{equation*}
        \forall y\in G, \mathrm{mat}_{\chi_1,\dots, \chi_n}(T_y) = \begin{pmatrix}
            \chi_1(y) & 0 & \dots & 0\\
            0 & \chi_2(y) & \dots & 0\\
            \vdots & \vdots & \ddots & \vdots\\
            0 & 0 & \dots & \chi_n(y)
        \end{pmatrix}.
    \end{equation*}
    \item Montrer que \(\widehat{G} = \{\chi_1,\dots,\chi_n\}\) et que 
    \((\widehat{G},\times)\) forme un groupe abélien. En déduire que 
    \((\widehat{G},\times)\) est isomorphe à \((G,+)\).
    \item Montrer que \(\chi_1,\dots,\chi_n\) forment une base orthonormale de \(E\).
    \item Montrer que pour tout 
    \begin{equation*}
        \sum_{\chi \in \widehat{G}} \chi(x) = \begin{cases}
            n & \text{ si } x = 0\\
            0 & \text{ sinon}.
        \end{cases}
    \end{equation*}
\end{enumerate}
\subsection{Caractères de \(\Z/n\Z\)}
\begin{enumerate}
    \item Montrer que l'endormorphisme \(\tau_1\) peut-être représenté par la matrice 
    \begin{equation*}
        T = \begin{pmatrix}
            0 & 1 & 0 & \dots & 0\\
            0 & 0 & 1 & \dots & 0\\
            \vdots & \vdots & \ddots & \vdots\\
            0 & 0 & \dots & 0 & 1\\
            1 & 0 & \dots & 0 & 0
        \end{pmatrix}.
    \end{equation*}
    \item Montrer que les valeurs propres de \(T\) sont les racines \(n\)-ièmes de l'unité.
    En déduire l'ensemble des caractères de \(\Z/n\Z\).
    \item En déduire qu'on peut indexer les caractères \(\chi_i\) de \(\Z/n\Z\) par des 
    \(l\in \Z/n\Z\) de sorte que 
    \begin{equation*}
        \forall k\in \Z/n\Z, \chi_l\chi_{-l'} = \chi_{l-l'}.
    \end{equation*}
    En conclure que \(\widehat{\Z/n\Z}\) est isomorphe à \(\Z/n\Z\).
    \subsection{Caractère d'un produit de groupes}
    Dans cette partie, on considère un groupe \(G = G_1\times \dots G_k\) tel que 
    \(\mathrm{card}(G) = n\).
    \begin{enumerate}
        \item Soient \(\chi^{(1)},\dots,\chi^{(k)}\) des caractères de \(G_1,\dots, G_k\).
        Montrer que \(\chi^{(1)}\times\dots\times\chi^{(k)} \) est un caractère de \(G\).
        \item Montrer alors que 
        \begin{equation*}
            \widehat{G} = \{\chi^{(1)}\times\dots\times\chi^{(k)},\;\; \chi^{(1)}\in \widehat{G_1},\dots, \chi^{(k)}\in \widehat{G_k}\}.
        \end{equation*}
        \item En déduire que \(\widehat{G}\) est isomorphe à \(G\).
        \item On considère le groupe \(G = (\Z/2\Z)^n\), montrer que les caractères de \(G\)
        sont de la forme 
        \begin{equation*}
            \chi_x(y) = (-1)^{\sum_{i=1}^n x_iy_i}.
        \end{equation*}
    \end{enumerate}
\end{enumerate}
On admettra désormais que le groupe dual \(\widehat{G}\) est 
toujours isomorphe à \(G\) dans le cas des groupes abéliens finis. On notera 
\(\chi_x\) le caractère associé à \(x\in G\). 

\section{Transformée de Fourier}
Pour une fonction \(f\in E\), on définit sa transformée de Fourier \(\hat{f}\) par 
\begin{equation*}
    \forall x\in G, \hat{f}(x) = n\braket{f}{\chi_x} = \sum_{y\in G} f(y)\overline{\chi_x(y)}.
\end{equation*}
On définit également la transformée de Fourier inverse par
\begin{equation*}
    \forall x\in G, \tilde{f}(x) = \frac{1}{n}\sum_{y\in G} f(y)\chi_y(x).
\end{equation*}
\begin{enumerate}
    \item Montrer que \(f\mapsto \hat{f}\) est une application linéaire de \(E\) dans \(E\).
    \item Montrer que pour tout \(f\in E\), on a \(\tilde{\hat{f}} = f\).
    \item Montrer que pour tout \(f\in E\), 
    \begin{equation*}
        \sum_{x\in G} \lvert \hat{f}(x)\rvert^2 = n\sum_{x\in G} \lvert f(x)\rvert^2.
    \end{equation*}
    \item Montrer que pour tout \(f,g\in E\), on a
    \begin{equation*}
        \widehat{f\star g} = \hat{f}\hat{g}.
    \end{equation*}
\end{enumerate}

\subsection*{Correction de la troisième partie}
\begin{enumerate}
    \item On vérifie les propriétés.
    \item On peut le vérifier de plusieurs manières. La première est de vérifier que l'image de 
    d'une base orthonormale est encore une base orthonormale. On prend, par exemple, la base des 
    \(\{\delta_x,x\in G\}\). 
    \begin{equation*}
        \braket{\tau_y\delta_x}{\tau_y\delta_z} = \braket{\delta_{x-y}}{\delta_{z-y}} = \delta_{x-y,z-y} = \delta_{x,z}.
    \end{equation*}
    Ce qui permet de conclure que \(\tau_y\) sera unitaire. Sinon, on peut aussi vérifier que l'inverse 
    de \(T_y\) est \(\overline{T_y^T}\). On a \(T_y^{-1} = T_{-y}\). On ensuite vérifier 
    pour chaque \(\delta_x\) que \(\overline{T_y^T} = T_{-y}\).
    \item On considère un morphisme \(\chi\). On a par définition 
    \begin{equation*}
        \forall x\in G, \;\tau_y(\chi)(x) = \chi(x+y) = \chi(y)\chi(x).
    \end{equation*}
    Donc \(\chi\) est un vecteur propre de \(\tau_y\) associé à la valeur propre \(\chi(y)\).
    \item On utilise le fait que \(\tau_y^n = \tau_{ny} = \tau_{0} = \mathrm{id}\). On en 
    déduit que \(X^n -1\) est annulateur de \(\tau_y\). Donc \(\tau_y\) est diagonalisable car 
    il est annulé par un polynôme scindé à racines simples.
    \item On sait qu'il existe alors une base commune de diagonalisation pour tous les \(y\) de \(\C^G\) formée de vecteurs propres de \(\tau_y\), on les note 
    \(\chi_1,\dots,\chi_n\). On vérifie que \(\chi_i\in \widehat{G}\) simplement. 
    \item On a vu qu'il y a au moins autant de valeur propres distinctes que de morphismes de groupe.
    Le résultat précédent montre que l'ensemble des valeurs propres est inclus dans \(\widehat{G}\).
    On en déduit que \(\widehat{G} = \{\chi_1,\dots,\chi_n\}\). On montre simplement la structure de groupe de \(\widehat{G}\).
    \item Les \(\chi_i\) forment une base car ce sont des vecteurs propres. Et d'après la partie préliminaire, 
    ces vecteurs sont orthogonaux. Il reste à montrer que ces vecteurs sont de norme 1.
    \begin{align*}
        \braket{\chi_i}{\chi_i} &= \frac{1}{n}\sum_{x\in G} \chi_i(x)\overline{\chi_i(x)}\\
        &= \frac{1}{n}\sum_{x\in G} \lvert\chi(x)\rvert^2\\
        &= 1.
    \end{align*}
    car les \(\chi_i(x)\) sont les valeurs propres de \(\tau_x\) qui est unitaire (donc les
    valeurs propres sont de module 1).
    \item Notons \(S(x) = \sum_{\chi\in \widehat{G}}\chi(x)\). Si \(x = 0\), on a facilement
    \(S(0) = n\). Sinon, on peut utiliser le fait que \(\chi \mapsto \chi' \chi\) soit 
    une bijection de \(\widehat{G}\) sur lui-même. On a
    \begin{align*}
        S(x) &=  \sum_{\chi\in \widehat{G}}\chi(x)\\
        &= \sum_{\chi\in \widehat{G}}\chi'(x)\chi(x)\\
        &= \chi'(x)\sum_{\chi\in \widehat{G}}\chi(x)\\
        & = \chi'(x)S(x).
    \end{align*}
    Or, pour tout \(x\neq 0\), il existe un \(\chi'\) tel \(\chi'(x) \neq 1\). On en conclut 
    que \(S(x) = 0\).
\end{enumerate}
\subsubsection*{Correction des questions sur \(\Z/n\Z\)}
\begin{enumerate}
    \item Si on considère à la base \(\{\delta_i\}\), on a \(\tau_1(\delta_i) = \delta_{i-1}\) pour \(i\) 
    différent de \(0\), et pour \(i=0\) on a \(\tau_1(\delta_{0}) = \delta_{n-1}\). On a donc
    \begin{equation*}
        \mathrm{mat}_{\delta_0,\dots,\delta_{n-1}}(\tau_1) = \begin{pmatrix}
            0 & 1 & 0 & \dots & 0\\
            0 & 0 & 1 & \dots & 0\\
            \vdots & \vdots & \ddots & \vdots\\
            0 & 0 & \dots & 0 & 1\\
            1 & 0 & \dots & 0 & 0
        \end{pmatrix}.
    \end{equation*}
    \item Un calcul de valeur propre...
    \item Il suffit d'associer à chaque \(l\in \Z/n\Z\) le caractère \(\chi_l\) défini par
    \begin{equation*}
        \forall k\in \Z/n\Z, \chi_l(k) = \exp\left(\frac{2i\pi lk}{n}\right).
    \end{equation*}
    On vérifie alors que \(\chi_l\chi_{-l'} = \chi_{l-l'}\). \(\chi(l) = 1\) si et seulement 
    si \(l=0\), donc \(l\mapsto \chi_l\) est un isomorphisme de \(\Z/n\Z\) vers \(\widehat{\Z/n\Z}\).
\end{enumerate}
    \subsubsection*{Correction des questions sur les groupes produits}
    \begin{enumerate}
        \item On vérifie les propriétés. On considère \((0,...,0)\in G\), on a 
        \begin{equation*}
            \chi^{(1)}\times\dots\times\chi^{(k)}(0,...,0) = \chi^{(1)}(0)\times\dots\times\chi^{(k)}(0) = 1.
        \end{equation*}
        \begin{align*}
            \chi^{(1)}\times\dots\times\chi^{(k)}((x_1,...,x_k)-(y_1,...,y_k)) &= \chi^{(1)}(x_1-y_1)\times\dots\times\chi^{(k)}(x_k-y_k)\\
            &= \chi^{(1)}(x_1)\times\dots\times\chi^{(k)}(x_k)\chi^{(1)}(-y_1)\times\dots\times\chi^{(k)}(-y_k)\\
            &= \chi^{(1)}(x_1)\times\dots\times\chi^{(k)}(x_k){\chi^{(1)}(y_1)}^{-1}\times\dots\times{\chi^{(k)}(y_k)}^{-1}\\
        \end{align*}
        Donc \(\chi^{(1)}\times\dots\times\chi^{(k)}\) est un caractère de \(G\).
        \item \(\chi^{(1)}\times \dots\times  \chi^{(k)}(x_1,\dots, x_k) = \rho^{(1)}\times \dots\times \rho^{(k)}(x_1,\dots, x_k)\).
        On peut appliquer, on note \(n_i\) le cardinal de \(G_i\), et on a pour tout \(j\)
        \begin{equation*}
            \chi^{(1)}\times \dots\times  \chi^{(k)}(n_1x_1,\dots,x_j,\dots, n_kx_k) = \rho^{(1)}\times \dots\times  \rho^{(k)}(n_1x_1,\dots,x_j,\dots, n_kx_k).
        \end{equation*}
        On en déduit que \(\chi_j(x_j) = \rho_j(x_j)\). Autrement dit, \(\prod_j \chi^{(j)} = \prod_j \rho^{(j)}\)
        si et seulement si \(\chi^{(j)} = \rho^{(j)}\). Ainsi, les \(\prod_j \chi^{(j)}\) sont distincts et forment 
        \(n_1\dots n_k = n\) morphismes de \(G\), donc \(\widehat{G} = \{\chi^{(1)}\times\dots\times\chi^{(k)},\;\; \chi^{(1)}\in \widehat{G_1},\dots, \chi^{(k)}\in \widehat{G_k}\}\)
    \end{enumerate}
\subsection*{Correction de la quatrième partie}
\begin{enumerate}
    \item 
\end{enumerate}
\section{Marche aléatoire}
On considère \((X^{(k)})_{k\in \N^*}\) une suite de variables aléatoires indépendantes
et indentiquement distribuées sur \(G\). On définit la marche aléatoire \(S_k\) par 
\begin{equation*}
    S_k = \sum_{l=1}^k X^{(l)}.
\end{equation*}
On notera \(p^{(k)}(x) = \P(S_k = x)\).
\begin{enumerate}
    \item Montrer que pour tout \(k\in \N^*\), on a
    \begin{equation*}
        \overbrace{p\star \dots \star p}^k = p^{(k)}.
    \end{equation*}
    Une première question qu'on se posera sera de savoir si \(S_k\) se rapproche
    de la loi uniforme \(\upsilon\) sur \(G\) pour la norme \(\ell^1\), i.e. étudier la 
    quantité suivante 
    \begin{equation*}
        \norm{p^{(k)} - \upsilon}_1 = \sum_{x\in G} \left\lvert p^{(k)}(x) - \frac{1}{n}\right\rvert.
    \end{equation*}
    \begin{enumerate}
        \item Montrer que pour tout \(k\in \N^*\), on a 
        \begin{equation*}
            \norm{p^{(k)}-\upsilon}_1 \le \sqrt{n} \norm{p-\upsilon}_2.
        \end{equation*}
        En déduire que pour tout \(k\in \N^*\), on a
        \begin{equation*}
            \norm{p^{(k)}-\upsilon}_1 \le \left\|\hat{p}^k - \widehat{\upsilon}\right\|_2
        \end{equation*}
        \item Montrer que pour tout \(k\in \N^*\), on a 
        \begin{equation*}
            \left\|\hat{p}^k - \widehat{\upsilon}\right\|_2^2 = \sum_{x\in G\setminus\{0\}} \lvert \hat{p}(x)\rvert^{2k}.
        \end{equation*} 
        \item On considère désormais que \(G = \Z/n\Z\) avec \(n\) impair et \(p(X=-1) = p(X=1) = \frac{1}{2}\).
        Montrer que pour tout \(z\in [0,\pi/2[\), on a 
        \begin{equation*}
            \cos\left(\frac{2z\pi}{n}\right) \le e^{-\frac{z^2}{2}}
        \end{equation*}
        \item Pour tout \(k\in \N^*\), montrer que 
        \begin{equation*}
            \sum_{x=1}^{n-1}\cos\left(\frac{2\pi x}{n}\right)^{2k} = 2\sum_{x=1}^{\frac{n-1}{2}}\cos\left(\frac{2\pi x}{n}\right)^{2k} = 2\sum_{x=1}^{\frac{n-1}{2}}\cos\left(\frac{\pi x}{n}\right)^{2k}
        \end{equation*}
        En déduire l'inégalité suivante
        \begin{equation*}
        \norm{p^{(k)}-\upsilon}_1^2 \le 2e^{-\frac{k\pi^2}{n^2}}\sum_{x=1}^{\frac{n-1}{2}}e^{-\frac{\pi^2(x^2-1)k}{n^2}}
        \end{equation*}
        \item Montrer que pour \(x\in\{1,\dots,n-1\}\), \((x^2-1)\ge 3(x-1)\), et en déduire que 
        \begin{equation*}
            \norm{p^{(k)}-\upsilon}_1^2 \le 2\frac{e^{-\frac{k\pi^2}{n^2}}}{1- e^{-\frac{3k\pi^2}{n^2}}}
        \end{equation*}
    \end{enumerate}
\subsection*{Correction de la cinquième partie}
\begin{enumerate}
    \item Généralement, on considère deux variables aléatoires indépendantes \(X\) et \(Y\). 
    La loi de \(X+Y\) sera donné par la convolution des lois de \(X\) et \(Y\). On a 
    \begin{align*}
        \P(X+Y = x) &= \sum_{y\in G} \P(X+Y = x,Y = y)\\
        &= \sum_{y\in G} \P(X = x-y,Y = y)\\
        &= \sum_{y\in G} \P(X = x-y)\P(Y = y).
    \end{align*}
    On fait ensuite la preuve par récurrence. \(p^{(1)} = p\). On suppose que 
    \(\overbrace{p\star \dots \star p}^k = p^{(k)}\). On a \(S_{k+1} = S_k + X^{(k+1)}\) avec 
    \(S_k\) et \(X^{(k+1)}\) indépendants. On a donc
    \begin{equation*}
        p^{(k+1)} = p^{(k)}\star p = \overbrace{p\star \dots \star p}^{k+1}.
    \end{equation*}
    \begin{enumerate}
        \item On utilise l'inégalité de Cauchy-Schwarz pour passer de la norme 1 à la norme 2.
        On utilise ensuite l'égalité de Plancherel.
        \item On remarque que \(\upsilon = \chi_0\) et \(\hat{p^{(k)}}(0) = n\)
        \begin{align*}
        \left\|\hat{p}^k - \widehat{\upsilon}\right\|_2^2 &= \sum_{x\in G} \lvert \hat{p}^k(x) - \widehat{\upsilon}(x)\rvert^2\\
        &= \sum_{x\in G}\left\lvert \hat{p^{(k)}}(x) - n\braket{\chi_0}{\chi_x} \right\rvert^2\\
        & = \sum_{x\in G\setminus\{0\}} \lvert \hat{p}(x)\rvert^{2k}.
        \end{align*}
        \item On commmence par calculer \(\hat{p}(x)\). On a 
        \begin{align*}
            \hat{p}(x) &= \sum_{y\in G} p(y)\overline{\chi_x(y)}\\
            &= \frac{1}{2}\left(\chi_{-1}(x) + \chi_1(x)\right)\\
            &= \frac{1}{2}\left(e^{-\frac{2x\pi}{n}}+ e^{\frac{2x\pi}{n}}\right)\\
            &= \cos\left(\frac{2\pi x}{n}\right)
        \end{align*}
        On a donc 
        \begin{equation*}
            \norm{p^{(k)}-\upsilon}^2 \le \sum_{x = 1}^{n-1}\cos\left(\frac{2\pi x}{n}\right)^{2k}.
        \end{equation*}
    \end{enumerate}
\end{enumerate}
\end{document}