\documentclass[a4paper]{article}
\usepackage{subfiles}
\usepackage{amsmath, amssymb, amsthm}
\usepackage{mdframed}
\usepackage{tikz}
\usepackage{tikz-cd}
\usepackage{hyperref}

\usepackage{mdframed}

\newcommand*{\R}{\mathbb{R}}
\newcommand*{\C}{\mathbb{C}}
\newcommand*{\N}{\mathbb{N}}
\newcommand*{\Z}{\mathbb{Z}}
\newcommand*{\Q}{\mathbb{Q}}
\newcommand*{\F}{\mathbb{F}}
\newcommand*{\Pb}{\mathbb{P}}
\newcommand*{\K}{\mathbb{K}}


\newmdtheoremenv{theorem}{Théorème}
\newmdtheoremenv{lemma}{Lemme}
\newmdtheoremenv{corollary}{Corollaire}
\newmdtheoremenv{definition}{Définition}
\newmdtheoremenv{proposition}{Proposition}
\newmdtheoremenv{remark}{Remarque}
\newmdtheoremenv{example}{Exemple}

\begin{document}
\author{Maxence Jauberty}
\title{Développements}
\maketitle
\begin{abstract} Ce document contient des développements mathématiques faits pour me 
    faire réviser de nombreuses notions, dans un but de préparer une agrégation.
\end{abstract}
\tableofcontents
\subfile{section/Demi-plan-de-Poincaré}
\newpage
\subfile{section/Nombre-moyen-de-points-fixes.tex}
\newpage
\subfile{section/Lemme-de-Zolotarev.tex}
\newpage
\subfile{section/Matrices-stochastiques.tex}
\newpage
\subfile{section/mobius-et-poly-irr.tex}
\newpage
\subfile{section/borel-cantelli-premier.tex}
\newpage
\subfile{section/composantes-connexes-GLnR.tex}
\newpage
\subfile{section/Critere-deisenstein.tex}
\newpage
\subfile{section/anneau-principal-non-euclidien.tex}

\end{document}