\documentclass[../main.tex]{subfiles}
\begin{document}
\section{Lemme de Zolotarev}
Si \(X\) est un ensemble fini quelconque, nous notons \(\mathfrak{S}_X\) le groupe des permutations de \(X\).
Nous rappelons qu'il existe un unique morphisme surjectif \(\epsilon : \mathfrak{S}_X \longrightarrow \{-1,1\}\)
appelé signature. 
\begin{theorem} Il existe un unique morphisme signature de \(\mathfrak{S}_X\).
\end{theorem}
\begin{proof}
    On commence par montrer l'existence. On note \(n = \lvert X \rvert\) et on considère l'application suivante
    \begin{equation}
        \Delta : \begin{cases}
            \mathbb{Z}^n & \longrightarrow \{-1,1\}\\
            (x_1,\ldots,x_n) & \longmapsto \prod_{1\le i<j \le n}^n x_j -x_i.
        \end{cases}
    \end{equation}
    On peut faire agir \(\mathfrak{S}_n\) sur l'ensemble des applications \(\mathbb{Z}^n\to \mathbb{Z}\) par l'action suivante
    \begin{equation}
        \sigma \cdot f (x_1, \ldots, x_n) = f(x_{\sigma(1)},\ldots, x_{\sigma(n)}).
    \end{equation}
    On remarque ensuite que l'action d'une transposition sur \(\Delta\) est d'en changer le signe.
    \begin{lemma}
        Soit \(\tau\) une transposition, alors \(\tau \cdot \Delta = -\Delta\).
    \end{lemma}
    On peut ensuite définir le morphisme \(\epsilon_n\) comme le signe de \(\Delta\).
    Celui-ci est bien un morphisme car pour une transposition \(\tau\), on a 
    \begin{equation}
        \epsilon_n(\sigma\tau) = \epsilon_n(\sigma)\epsilon_n(\tau) = -\epsilon_n(\sigma).
    \end{equation}
    Or \(\mathfrak{S}_n\) est engendré par les transpositions, donc \(\epsilon_n\) est bien un morphisme de groupes.
    Par ailleurs, il est bien surjectif puisque \(\epsilon_n(\tau) = -1\).
    \\
    Il existe un isomorphisme de groupes entre \(\mathfrak{S}_X\) et \(\mathfrak{S}_n\). On note \(f\) un isomorphisme entre ces deux groupes.
    On peut donc définir \(\epsilon = f\circ \epsilon_n\) qui est un morphisme signature.
    \\

    Montrons que ce morphisme est unique. Soit \(\epsilon'\) un autre morphisme signature non trivial, i.e. \(\epsilon'(\tau) = -1\).
    Puisque \(\mathfrak{S}_X\) est engendré par les transpositions, il existe pour \(\sigma\in \mathfrak{S}_X\) une suite de transpositions \(\tau_1,\ldots, \tau_k\) telles que \(\sigma = \tau_1\ldots \tau_k\).
    On a alors
    \begin{equation}
        \epsilon'(\sigma) = \epsilon'(\tau_1)\ldots \epsilon'(\tau_k) = (-1)^k = \epsilon(\sigma).
    \end{equation}
\end{proof}
Le lemme de Zolotarev est une conséquence directe de ce théorème.
\begin{theorem} Soit \(p\) premier et \(a\in \Z/p\Z^\times\), on définit \(\mathfrak{m}_a\) comme la multiplication par \(a\) dans \(\Z/p\Z^\times\), i.e.
    \begin{equation}
        \mathfrak{m}_a : \begin{cases}
            \Z/p\Z^\times & \longrightarrow \Z/p\Z^\times\\
            x & \longmapsto ax.
        \end{cases}
    \end{equation}
    Alors, \(\mathfrak{m}_a\) est une permutation de \(\Z/p\Z^\times\) de signature \(\epsilon(\mathfrak{m}_a) = \left(\frac{a}{p}\right)\).
\end{theorem}
\begin{proof} \(\mathfrak{m}_a\) est une permutation car \(\Z/p\Z\) est un corps, la bijection inverse est alors \(\mathfrak{m}_{a^{-1}}\).

    On peut ensuite remarquer que le symbole de Legendre est un morphisme de groupes de \(\Z/p\Z^\times\) dans \(\{-1,1\}\) non trivial et surjectif.
    Cela conclut la preuve : \(\epsilon(\mathfrak{m}_a) = \left(\frac{a}{p}\right)\).
\end{proof}
Le lemme de Zolotarev permet de faire un pont entre la théorie des nombres et la théorie des permutations.
Si on considère, par exemple, \(\mathfrak{m}_2\) peut se représenter comme la permutation suivante
\begin{equation}
    \begin{pmatrix}
        1 & 2 &  \ldots & \frac{p-1}{2} & \frac{p+1}{2}  & \ldots & p-2 & p-1\\
        2 & 4 &  \ldots & p-1 & p+1 & \ldots & p-4 & p-2
    \end{pmatrix}.
\end{equation}
On remarque que l'inversion se fait à partir de \(\frac{p+1}{2}\) et \(\epsilon(\mathfrak{m}_2) = (-1)^{\text{nb d'inversion}}\) donc
\begin{equation}
    \epsilon(\mathfrak{m}_2) = (-1)^{\frac{p-1}{2}+\ldots + 2 + 1}.
\end{equation}
Il ne suffit que de calculer, \(\frac{p-1}{2}+\ldots + 2 + 1 = \frac{p^2-1}{8}\), d'où
\begin{equation}
    \left(\frac{2}{p}\right) = \epsilon(\mathfrak{m}_2) = (-1)^{\frac{p^2-1}{8}}.
\end{equation}
\subsection*{Une application}
On peut utiliser le lemme de Zolotarev pour démontrer la loi de réciprocité quadratique. 
\begin{theorem} Soit \(p\) et \(q\) deux nombres premiers impairs, alors
    \begin{equation}
        \left(\frac{p}{q}\right)\left(\frac{q}{p}\right) = (-1)^{\frac{p-1}{2}\frac{q-1}{2}}.
    \end{equation}
\end{theorem}
\begin{proof} D'abord, nous considérons l'isomorphisme \(\phi : \Z_{pq}\longrightarrow \Z_p \times \Z_q\).
    On considère les deux permutations suivantes 
    \begin{align}
        \sigma &: \begin{cases}
            \Z_p \times \Z_q & \longrightarrow \Z_p \times \Z_q\\
            (x,y) & \longmapsto (qx + y,y)
        \end{cases}\\
        \tau &: \begin{cases}
            \Z_p \times \Z_q & \longrightarrow \Z_p \times \Z_q\\
            (x,y) & \longmapsto (x,py + x)
        \end{cases}.
    \end{align}
    On définit ensuite \(\rho\) sur \(\Z_{pq}\) par \(\rho(x+qy) = px+y\). On peut alors remarquer que
    \begin{equation}
        \phi(qx+y) = (qx+y,y) = \sigma(x,y) \text{ et } \phi(px+y) = (x,py+x) = \tau(x,y).
    \end{equation}
    Ainsi, \(qx+y = \phi^{-1}(\sigma(x,y))\) et en appliquant \(\rho\), on obtient 
    \(\rho(\phi^{-1}(\sigma(x,y))) = px+y\). Finalement, en appliquant \(\phi\), déduit l'égalité suivante
    \begin{equation}
        \phi\circ \rho \circ \phi^{-1} \circ \sigma = \tau.
    \end{equation}
    Désormais, on peut considère les signatures de ces permutations.
    \begin{equation}
        \epsilon(\phi\circ \rho \circ \phi^{-1})\epsilon(\sigma) = \epsilon(\tau).
    \end{equation}
    On sait que \(\epsilon(\sigma) = \left(\frac{q}{p}\right)\) et \(\epsilon(\tau) = \left(\frac{p}{q}\right)\).
    (A faire) De plus, \(\epsilon(\phi\circ \rho \circ \phi^{-1}) = (-1)^{\frac{(p-1)(q-1)}{4}}\), d'où la loi de réciprocité 
    \begin{equation}
        \left(\frac{p}{q}\right)\left(\frac{q}{p}\right) = (-1)^{\frac{p-1}{2}\frac{q-1}{2}}.
    \end{equation}
\end{proof}
\subsection*{Une généralisation}
\begin{theorem} Soit \(E\) un espace vectoriel sur un corps fini \(\F_p\), alors pour tout
    automorphisme \(u\) de \(E\), on a
    \begin{equation}
        \epsilon(u) = \left(\frac{\det(u)}{p}\right).
    \end{equation}
    où \(\epsilon\) est le morphisme signature de \(GL(E)\).
\end{theorem}
\begin{proof} Remarquons que \(SL(E)\) est le groupe dérivé de \(GL(E)\). On en déduit que
    tout morphisme de \(GL(E)\) vers un groupe abélien se factorise à travers l'abélianisé \(GL(E)/SL(E)\).
    En particulier, si nous considérons le morphisme signature de \(\mathfrak{S}_E\), nous avons l'existence
    de \(f : \F_p^\times \longrightarrow \{-1,1\}\) tel que \(\epsilon = f\circ \det\) sur \(GL(E)\).

    De plus, \(f\) n'est pas un morphisme trivial, on peut, par exemple, prendre la multiplication par
    un élément générateur de \(\F_p^\times\). Cet automorphisme est une permutation circulaire, ce qui permet simplement de déterminer que sa signature est \(-1\).
    On en déduit que \(f\) n'est pas le morphisme trivial, donc \(f = \left(\frac{.}{p}\right)\) par unicité du symbole de Legendre.
    D'où 
    \begin{equation}
        \epsilon(u) = \left(\frac{\det(u)}{p}\right).
    \end{equation}
\end{proof}

\end{document}