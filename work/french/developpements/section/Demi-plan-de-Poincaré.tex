\documentclass[..\main.tex]{subfiles}

\begin{document}
\section{Demi-plan de Poincaré}

\begin{definition}
    On appelle \emph{demi-plan de Poincaré} l'ensemble suivant
    \begin{equation}
        \mathcal{H} = \{ z \in \mathbb{C} \mid \Im(z) > 0 \}\cup \{\infty\}.
    \end{equation}
\end{definition}
Sur le demi-plan de Poincaré, les droites, ou plus exactement les géodésiques, 
sont définies comme les demi-cercles dont le centre est sur l'axe des réels et
les droites verticales, i.e. les droites passant par $\infty$.
\begin{figure}
    \begin{tikzpicture}
        \draw[->] (-2,0) -- (4,0);
        \draw[->] (0,-2) -- (0,4);
        \draw[red] (1,-2) -- (1,4);
        \draw[blue] (3,0) arc (3:180:1);
    \end{tikzpicture}
\end{figure}
\end{document}