\documentclass[../main.tex]{subfiles}
\begin{document}
\section{Composantes connexes de \(\mathcal{GL}_n(\R)\)}
\begin{abstract}
    On montre que 
\end{abstract}
\begin{definition}
    Une partie d'un espace topologique \(X\) est dite connexe par arcs si pour tout \(x,y\) dans \(X\), il existe
    une application continue \(\gamma:[0,1]\to X\) telle que \(\gamma(0)=x\) et \(\gamma(1)=y\).
    \begin{remark} On appelera \(\gamma\) un chemin de \(x\) à \(y\) dans \(X\). Sur le dessin, cela revient 
        à montrer que l'on peut \textit{tracer} un chemin continu n'importe quel couple de points.
    \end{remark}
\end{definition}
On pourra également considérer la relation d'équivalence \(\longleftrightarrow\) sur \(X\) définie par
\begin{equation}
    x\longleftrightarrow y \iff \exists \gamma:[0,1]\to X \text{ telle que } \gamma(0) = x \text{ et } \gamma(1) = y.
\end{equation}
Pour un \(x\in X\) donné, on pourra appelé \(\mathcal{C}(x)\) la composante connexe par arcs de \(x\).
Dire que \(X\) est connexe par arcs est équivalent à dire que pour un \(x\in X\), \(\mathcal{C}(x) = X\).\\

\begin{lemma} Si \(f\) est une application continue de \(X\) dans \(Y\), deux espaces topologiques, et que
    \(X\) est connexe par arcs, alors \(f(X)\) est connexe par arcs dans \(Y\).
\end{lemma}
\begin{proof}
    Soient \(f(x_1),f(x_2)\in f(X)\). \(X\) est connexe par arcs donc il existe \(\gamma\) un chemin de \(x_1\) vers \(x_2\).
    \(f\circ\gamma\) est alors un chemin de \(f(x_1)\) vers \(f(x_2)\).
\end{proof}
\begin{proposition} \(\mathcal{GL}_n(\R)\) n'est pas connexe par arcs.
\end{proposition}
\begin{proof} On remarquera que \(\det\) est une application linéaire surjective et continue de \(\mathcal{GL}_n(\R)\) dans \(\R^*\).
    Or, \(\R^*\) n'est pas connexe par arcs. Par contraposée du lemme précédent, \(\mathcal{GL}_n(\R)\) n'est pas connexe par arcs.
\end{proof}
Nous allons en fait montrer que \(\mathcal{GL}_n(\R)\) admet deux composantes connexes par arcs.
\begin{lemma} \(\mathcal{O}_n(\R)\) admet deux composantes connexes par arcs, \(\mathcal{O}_n^+\) et \(\mathcal{O}_n^-\).
\end{lemma}
Evidemment, \(\mathcal{O}_n(\R)\)  n'est pas connexe par arcs. Simplement car un espace discret ne peut être connexe par arcs que s'il est réduit à un singleton, ce qui n'est pas le cas de \(\det(\mathcal{O}_n(\R))\).
\begin{proof} On note \(R(\theta)\) une matrice de rotation \(\theta\).
    Soit \(A\in \mathcal{O}_n^+(\R)\). On sait qu'il existe une matrice orthogonale \(P\)
    telle que \(A = PDP^{-1}\) où \(D = \mathrm{Diag}(I_r,R(\pi),\ldots,R(\pi),R(\theta_1),\ldots,R(\theta_p))\).
    On considère \(\delta\) un chemin de \(\pi\) à \(0\). On pose \(\Delta(t) = \mathrm{Diag}(I_r,R(\delta(t)),\ldots,R(\delta(t)))\).
    Ensuite, on considère \(\gamma_{i}\) un chemin de \(\theta_i\) à \(0\) dans \(]-\pi,\pi[\).

    On finit par poser \(\Gamma(t) = P\mathrm{Diag}(\Delta(t),R(\gamma_1(t)),\ldots,R(\gamma_p(t)))P^{-1}\).
    On a \(\Gamma(0) = A\) et \(\Gamma(1)=I_n\). \(\theta\mapsto R(\theta)\) est continue, on en déduit que \(t\mapsto \Delta(t)\) et \(t\mapsto \Gamma(t)\) sont continues aussi.
    On conserve la forme de "diagonale de rotation" ce qui permet d'assurer que \(\Gamma\) est à valeurs dans \(\mathcal{O}_n(\R)\).
    \begin{equation}
        \det(\Gamma(t)) = \underbrace{\det(\Delta(t))}_{=1\text{ ou } =(-1)^{2m}=1}\prod_{i=1}^p \underbrace{\det(R(\gamma_i(t)))}_{=1} = 1.
    \end{equation}
    On a donc trouvé un chemin de \(A\) à \(I_n\), ce qui suffit à montrer que \(\mathcal{O}_n^+(\R)\) est connexe par arcs.\\

    La même démonstration se fait en créant un chemin vers \(\mathrm{-1,I_{n-1}}\).\\

    Finalement, il ne suffit que de montrer que \(\mathcal{O}_n^+(\R)\) et \(\mathcal{O}_n^-(\R)\) forment une partition de \(\mathcal{O}_n(\R)\), ce qui est trivial.
    On a donc montré que \(\mathcal{O}_n(\R)\) admet deux composantes connexes par arcs, \(\mathcal{O}_n^+(\R)\) et \(\mathcal{O}_n^-(\R)\).
\end{proof}
\begin{lemma} On note \(\mathcal{S}^{++}_n(\R)\) l'ensemble des matrices symétriques définies
    postives. Alors \(\mathcal{S}^{++}_n(\R)\) est connexe par arcs.
\end{lemma}
\begin{proof} La preuve est plus simple. On considère \(A\in \mathcal{S}_n^{++}(\R)\).
    D'après le théorème spectral, il existe une matrice orthogonale \(P\) telle que
    \(A = PDP^{-1}\) avec \(D = \mathrm{Diag}(\lambda_1,\ldots,\lambda_n)\) et \(\lambda_i > 0\).
    (Réciproquement, une matrice d'une telle forme est symétrique définie positive).\\
    \(\R^*_+\) est connexe par arcs donc il existe un chemin \(\gamma_i\) de \(\lambda_i\) ) \(1\) dans \(\R^*_+\).
    On pose ensuite \(\Gamma(t) = P\mathrm{Diag}(\gamma_1(t),\ldots,\gamma_n(t))P^{-1}\).
    \(\Gamma\) est continue, à valeurs dans \(\mathcal{S}_n^{++}(\R)\) et \(\Gamma(0) = A\), \(\Gamma(1) = I_n\).
    On a donc trouvé un chemin de \(A\) à \(I_n\). \(\mathcal{S}_n^{++}(\R)\) est donc connexe par arcs.
\end{proof}
\begin{lemma}[Décomposition(s) polaire(s)]
    Soit \(A\in \mathcal{GL}_n(\R)\). Alors il existe \(O\in \mathcal{O}_n(\R)\) et \(S\in \mathcal{S}_n^{++}(\R)\) telles que
    \begin{equation}
        A = SO.
    \end{equation}
    De plus, si \(\det(A)>0\), alors \(O\in \mathcal{text{O}}_n^+(\R)\) et si \(\det(A)<0\), alors \(O\in \mathcal{O}_n^-(\R)\).
\end{lemma}
\begin{proof} \(A\) est inversible donc \(A^TA\) est symétrique définie positive.
    \begin{equation}
        A = AA^T{A^T}^{-1}.
    \end{equation}
    \(AA^T\) est symétrique définie positive donc il existe \(S\) symétrique définie positive telle que \(AA^T = S^2\).
    D'où \(A = SS{(A^T)}^{-1}\). En posant \(O = S{(A^T)}^{-1}\), on a bien \(A = SO\).\\
    Il reste à montrer que \(O\) est orthogonale. On a 
    \begin{equation}
        O^TO = {(S{(A^T)}^{-1})}^TS{(A^T)}^{-1} = A^{-1}SS{(A^T)}^{-1} = I_n.
    \end{equation}
    \begin{equation}
        OO^T = S{(A^T)}^{-1} {(S{(A^T)}^{-1})}^T = S{(A^T)}^{-1}A^{-1}S = I_n.
    \end{equation}\\

    Pour la remarque supplémentaire du lemme, on a \(\det(A) = \det(S)\det(O)\). Or, \(\det(S)>0\) et \(\det(O) = \det(A)\).
\end{proof}
\begin{theorem}
    \(\mathcal{GL}_n(\R)\) admet deux composantes connexes par arcs, \(\mathcal{O}_n^+(\R)\mathcal{S}_n^{++}(\R)\) et \(\mathcal{O}_n^-(\R)\mathcal{S}_n^{++}(\R)\).
\end{theorem}
\begin{proof} Le théorème est une utilisation de tous ces lemmes. Soit \(A\in \mathcal{GL}_n(\R)^+\), on considère
    sa décomposition polaire \(S,O\), donc \(O\in \mathcal{O}_n^+(\R)\) et \(S\in \mathcal{S}_n^{++}(\R)\). On note
    \(\Theta\) un chemin de \(O\) à \(I_n\) dans \(\mathcal{O}_n^+(\R)\) et \(\Sigma\) un chemin de \(S\) à \(I_n\) dans \(\mathcal{S}_n^{++}(\R)\). 
    On pose \(\Gamma(t) = \Theta(t)\Sigma(t)\). \(\Gamma\) est continue, à valeurs dans \(\mathcal{GL}_n(\R)^+\) et \(\Gamma(0) = A\), \(\Gamma(1) = I_n\).
    On a donc trouvé un chemin de \(A\) à \(I_n\). On a donc montré que \(\mathcal{GL}_n(\R)^+\) est connexe par arcs.\\

    On fait la même preuve pour \(\mathcal{GL}_n(\R)^-\) en créant un chemin vers \(\mathrm{Diag}(-1,I_{n-1})\).\\

    Puisque \(\mathcal{GL}_n(\R)^+\) et \(\mathcal{GL}_n(\R)^-\) forment une partition de \(\mathcal{GL}_n(\R)\), on a montré que \(\mathcal{GL}_n(\R)\) admet deux composantes connexes par arcs.
\end{proof}
\end{document}