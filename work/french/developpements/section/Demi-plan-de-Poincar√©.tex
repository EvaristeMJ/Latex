\documentclass[..\main.tex]{subfiles}

\begin{document}
\section{Demi-plan de Poincaré}

\begin{definition}
    On appelle \emph{demi-plan de Poincaré} l'ensemble suivant
    \begin{equation}
        \mathcal{H} = \{ z \in \mathbb{C} \mid \Im(z) > 0 \}\cup \{\infty\}.
    \end{equation}
\end{definition}
Sur le demi-plan de Poincaré, les droites, ou plus exactement les géodésiques, 
sont définies comme les demi-cercles dont le centre est sur l'axe des réels et
les droites verticales, i.e. les droites passant par $\infty$. On notera \(\mathcal{D}\)
l'ensemble de ces géodésiques.
\begin{figure}
    \begin{tikzpicture}
        \draw[->] (-2,0) -- (4,0);
        \draw[->] (0,-2) -- (0,4);
        \draw[red] (0.5,-2) -- (0.5,4);
        \draw[blue] (3,0) arc (0:180:1);
    \end{tikzpicture}
    \caption{Exemple de géodésiques sur le demi-plan de Poincaré, à noter que les points qui sont sur l'axe des réels sont exclus des géodésiques.}
\end{figure}
On considère l'ensemble des transformations projectives inversibles, soit l'ensemble \(PGL_2(\mathbb{R})\).
De telles transformations ont la propriété de "préserver" l'infini. Autrement dit, pour 
\(f \in PGL_2(\mathbb{R})\), on a \(f(\infty) = \infty\). Pour tout autre point de \(\mathbb{P}_1(\mathbb{C})\),
\(f\) peut être considérée comme une transformation linéaire. On parle alors de \emph{transformation de Moebius}.
\begin{lemma} Soient \(z,w\in \mathcal{H}\), il existe une unique droite géodésique de \(\mathcal{H}\) passant par \(z\) et \(w\).
\end{lemma}
\begin{proposition} \(PSL_2(\mathbb{R})\) agit sur \(\mathcal{H}\). De plus,
    il agit transitivement. 
\end{proposition}
\end{document}