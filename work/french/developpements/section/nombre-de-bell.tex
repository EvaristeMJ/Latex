\documentclass[../main.tex]{subfiles}
\begin{document}
\section{Nombres de Bell}
\begin{abstract} On note \(b_n\) le nombre de partitions d'un ensemble à \(n\) éléments.
    On montre que
    \begin{equation}
        b_n = \frac{1}{e}\sum_{k=0}^{\infty}\frac{k^n}{k!}.
    \end{equation}
\end{abstract}
Par convention, on pose \(b_0=1\). On pourrait débattre philosophiquement du nombre de partitions
du vide mais tout est plus simple avec \(b_0=1\).
\\
Le dénombrement de ces partitions se fera principalement par une étude de 
fonctions génératrices.
\begin{definition} Soit \(a\) une suite. On appelle fonction génératrice exponentielle de \(a\)
    la série formelle
    \begin{equation}
        A(x) = \sum_{n=0}^{\infty}\frac{a_n}{n!}x^n.
    \end{equation}
\end{definition}
Les fonctions génératrices sont des outils puissants. En particulier, elles caractérisent
complètement une suite, i.e.
\begin{equation}
    A = C \iff \forall n, a_n = c_n.
\end{equation}
\begin{lemma} Soit \(u,v\) deux suites et \(U,V\) leur FGE. On a 
    \begin{equation}
        U(x)V(x) = \sum_{n=0}^{\infty}\left(\sum_{k=0}^n \binom{n}{k} u_kv_{n-k}\right)\frac{x^n}{n!}.
    \end{equation}
\end{lemma}
\begin{proof}
    \begin{align}
        U(x)V(x) &= \sum_{n=0}^{\infty}\left(\sum_{k=0}^n \frac{u_k}{k!}\frac{v_{n-k}}{(n-k)!}\right)x^n\\
        & = \sum_{n=0}^{\infty}\left(\sum_{k=0}^n \frac{1}{n!}\frac{n!}{(n-k)!k!}u_{k}v_{n-k}\right)x^n\\
        & = \sum_{n=0}^{\infty}\left(\sum_{k=0}^n \binom{n}{k}u_{k}v_{n-k}\right)\frac{x^n}{n!}
    \end{align}
\end{proof}

Dans ce genre de dénombrement, on exhibe une relation de récurrence pour exprimer différemment
la fonction génératrice.
\begin{proposition}
    \begin{equation}
        b_{n+1} = \sum_{k=0}^n \binom{n}{k}b_k.
    \end{equation}
\end{proposition}
\begin{proof}
    On considère \(\{1,...,n+1\}\). Sans perte de généralité, on suppose que \(n+1\) est 
    dans une partition non-vide \(P\). On fixe ensuite \(k+1\) la taille de \(P\). Il y a
    \(k\) éléments à choisir parmi \(n\) pour former \(P\). Il reste \(n-k\) éléments à
    partitionner. On a donc \(b_{n-k}\) façons de le faire par définition. On somme
    sur \(k\) pour obtenir le résultat.
    \begin{equation}
        b_{n+1} = \sum_{k=0}^n \binom{n}{k}b_{n-k}.
    \end{equation}
    Il suffit de renverser la somme et d'utiliser la symétrie du binôme pour conclure.
\end{proof}
\begin{proposition} On a 
    \begin{equation}
        B'(x) = B(x)e^x.
    \end{equation}
\end{proposition}
\begin{proof}
    \begin{align}
        B'(x) & = \sum_{n=0}^{\infty}\frac{b_{n+1}}{n!}x^n\\
        & = \sum_{n=0}^{\infty}\frac{1}{n!}\sum_{k=0}^n \binom{n}{k}b_kx^n\\
    \end{align}
    Cela peut être vu comme une convolution de \(b\) avec la suite constante égale à \(1\).
    \begin{equation}
        B'(x) = B(x)\sum_{n=0}^{\infty}\frac{x^n}{n!} = B(x)e^x.
    \end{equation}
\end{proof}
\begin{theorem} On a 
    \begin{equation}
        b_n = \frac{1}{e}\sum_{k=0}^{\infty}\frac{k^n}{k!}.
    \end{equation}
\end{theorem}
\begin{proof}
    Il suffit de résoudre l'équation différentielle:
    \begin{equation}
        B'(x) = B(x)e^x, \quad B(0) = 1.
    \end{equation}
    La solution est unique et est donnée par
    \begin{equation}
        B(x) = \frac{1}{e}e^{e^x}.
    \end{equation}
    On peut ensuite écrire le terme de droite sous la forme d'une série exponentielle :
    \begin{align}
        e^{e^x} &= \sum_{n=0}^{\infty}\frac{1}{n!}(e^x)^n\\
        &= \sum_{n=0}^\infty \frac{1}{n!} \sum_{k=0}^\infty \frac{(xn)^k}{k!}\\
        & = \sum_{n=0}^\infty \sum_{k=0}^\infty \frac{n^k}{k!n!}x^k\\
        &= \sum_{k=0}^\infty \frac{1}{k!}\sum_{n=0}^\infty \frac{n^k}{n!}x^k.
        \end{align}
    Donc \(e^{e^x}\) est la série exponentielle de \(\sum_{k=0}^\infty \frac{k^n}{k!}\).
    On a donc, par égalité des fonctions génératrices,
    \begin{equation}
        b_n = \frac{1}{e}\sum_{k=0}^{\infty}\frac{k^n}{k!}.
    \end{equation}
\end{proof}
La présence d'une somme infinie dans la formule peut être considérée comme moralement gênante.
On peut cependant exprimer \(b_n\) à l'aide d'une somme finie. D'ailleurs, une première
manière de le faire est de calculer une approximation (pas trop mauvaise) de la somme puis de
prendre la partie entière. Avec l'inconvénient que l'on ne sait pas jusqu'où calculer.
On montre dans la prochaine proposition qu'il suffit de le faire jusqu'à \(2n-1\).
\begin{proposition} Soit \(n\ge 1\) un entier naturel. On a 
    \begin{equation}
        b_n = \lfloor\frac{1}{e}\sum_{k=0}^{2n-1}\frac{k^n}{n!}\rfloor + 1.
    \end{equation}
\end{proposition}
\begin{proof}
    On commence par montrer le lemme suivant.
    \begin{lemma} Soit \(n\ge 1\), on a
        \begin{equation}
            \left(1+\frac{1}{n}\right)^n< 2n+1.
        \end{equation}
    \end{lemma}
    \begin{proof}[du lemme]
        Pour \(n=1\), \(2<3\). On considère le résultat vrai pour \(n\),
        \(\left(1+\frac{1}{n+1}\right)^n < \left(1+\frac{1}{n}\right)^n < 2n+1\).
        On a 
        \begin{align}
            \left(1+\frac{1}{n+1}\right)^{n+1} &= \left(1+\frac{1}{n+1}\right)^n\left(1+\frac{1}{n+1}\right)\\
            &< (2n+1)\left(1+\frac{1}{n+1}\right)\\
            &= 2n+1+\frac{2n+1}{n+1}\\
            &= 2n+1 + 2 - \frac{1}{n+1}\\
            &< 2(n+1) + 1.
        \end{align}
    \end{proof}
    Remarquons ensuite que \((2n)^n\le (2n)!\). En effet, on a 
    \begin{equation}
        \frac{(2n)^n}{(2n)!} = \underbrace{\frac{2^n}{n!}}_{\le 2} \underbrace{\frac{n^n}{(n+1)\ldots 2n}}_{\le \frac{1}{2}} \le 1.
    \end{equation}
    On remarquera ensuite que \(\left(\frac{K^n}{K!}\right)_{K\ge 2n}\) est décroissante et donc que
    \begin{equation}
        \forall K \ge 2n, \frac{K^n}{K!} \le \frac{(2n)^n}{(2n)!} \le 1.
    \end{equation}
    On considère maintenant \(K \ge 2n\). On a pour \(j\ge 0\),
    \begin{equation}
        \frac{(K+j)^n}{(K+j)!}\le \frac{1}{j!}\frac{K^n}{K!}.
    \end{equation}
    En effet, remarquons que 
    \begin{align}
        \frac{(K+j)!}{K!j!} & = \frac{1}{K!}(K+j)\ldots (j+1)\\
        &= \left(\frac{K}{K} + \frac{j}{K}\right)\ldots \left(\frac{1}{1} + \frac{j}{1}\right)\\
        & = \left(1 + \frac{j}{K}\right)\ldots \left(1 + j\right)\\
        & \ge \left(1+\frac{j}{K}\right)^K \\
        &\ge \left(1+\frac{j}{K}\right)^n
    \end{align}
    On arrange alors
    \begin{equation}
        \frac{(K+j)^n}{K^n} \le \frac{(K+j)!}{j!K!}.
    \end{equation}
    Et finalement l'inégalité souhaitée. On déduit de nos résultats précédents :
    \begin{equation}
        \frac{(K+j)^n}{(K+j)!}\le \frac{1}{j!}\frac{K^n}{K!} \le \frac{1}{j!}
    \end{equation} On peut alors conclure notre résultat :
    \begin{align}
        b_n &= \frac{1}{e}\sum_{k=0}^\infty \frac{k^n}{k!}\\
        & = \frac{1}{e}\left(\sum_{k=0}^{K-1}\frac{k^n}{k!} + \sum_{k=K}^\infty \frac{k^n}{k!}\right)\\
        & = \frac{1}{e}\left(\sum_{k=0}^{K-1}\frac{k^n}{k!}+ \sum_{j=0}^\infty \frac{(K+j)^n}{(K+j)!}\right)\\
    \end{align}
    On a alors
    \begin{equation}
        0 < b_n - \frac{1}{e}\sum_{k=0}^{K}\frac{k^n}{k!} \le 1.
    \end{equation}
    D'où 
    \begin{equation}
        b_n = \left\lfloor\frac{1}{e}\sum_{k=0}^{K-1}\frac{k^n}{k!}\right\rfloor + 1.
    \end{equation}
    En particulier, on peut prendre \(K=2n-1\).
\end{proof}
\end{document}