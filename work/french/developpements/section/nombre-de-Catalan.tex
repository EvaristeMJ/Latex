\documentclass[../main.tex]{subfiles}
\begin{document}
\section{Nombre de Catalan}
\begin{abstract}
    On note \(c_n\) le nombre de parenthésages bien formés de longueur \(2n\), on parle aussi
    de chemins de Dyck. Ce nombre est appelé le \(n\)-ième nombre de Catalan. On montre que \(c_n = \frac{1}{n+1}\binom{2n}{n}\).
\end{abstract}
\begin{proposition} On a 
    \begin{equation}
        c_{n+1} = \sum_{k=0}^n c_kc_{n-k}
    \end{equation}
\end{proposition}
\begin{proof} On considère un parenthésage bien formé \(w\) de longueur \(2(n+1)\). On peut
    le décomposer en deux sous-parenthésages \(x\) et \(y\) de longueurs respectives \(2k\) et \(2(n-k)\) pour un certain \(k\).
    \begin{equation}
        w = (x)y
    \end{equation}
    Ainsi, pour \(k\) fixé, il y a \(c_k\) façons de choisir \(x\) et \(c_{n-k}\) façons de choisir \(y\). On a donc
    \(c_kc_{n-k}\) façons de choisir \(x\) et \(y\). En sommant sur \(k\), on obtient le résultat :
    \begin{equation}
        c_{n+1} = \sum_{k=0}^n c_kc_{n-k}
    \end{equation}
\end{proof}
On considère \(C(x) = \sum_{n=0}^\infty c_nx^n\).
\begin{proposition}
    \begin{equation}
        C(x) = 1 + xC(x)^2
    \end{equation}
\end{proposition}
\begin{proof}
    \begin{align}
        C(x) & = 1 + \sum_{n=1}^\infty c_nx^n\\
        & = 1 + x\sum_{n=0}^\infty c_{n+1}x^n\\
        &= 1 + x \sum_{n=0}^\infty \sum_{k=0}^n c_kc_{n-k}x^n\\
        & = 1 + x C(x)^2
    \end{align}
\end{proof}
Cela donne une equation du second degré pour \(C(x)\), on a 
\begin{equation}
    C(x) = \frac{1 + \epsilon(x) \sqrt{1-4x}}{2x}
\end{equation}
Remarquons que \(\epsilon\) n'est pas constante a priori. Cependant, \(\epsilon\) est 
continue et \(C(0)=-1\). Les seules valeurs de \(\epsilon\) sont \(1\) et \(-1\), le 
théorème des valeurs intermédiaires implique que \(\epsilon\) est constante. On en déduit
\begin{equation}
    C(x) = \frac{1 - \sqrt{1-4x}}{2x}.
\end{equation}
Il suffit ensuite d'exprimer le terme de droite en série entière.
\begin{proposition}
    \begin{equation}
        c_n = \frac{1}{n+1}\binom{2n}{n}
    \end{equation}
\end{proposition}
\begin{proof}
    On a 
    \begin{align}
        C(x) & = \frac{1 - \sqrt{1-4x}}{2x}\\
        & = -\frac{1}{2x}\sum_{n=1}^\infty \binom{1/2}{n}(-4x)^n\\
    \end{align}
    De plus, on a 
    \begin{align}
        \binom{1/2}{n} &= \frac{1/2(1/2-1)\ldots(1/2-n+1)}{n!}\\
        & = \frac{(-1)^{n-1}(2n-3)(2n-5)\ldots 1}{2^n n!}\\
        & = \frac{(-1)^{n-1}(2n-2)!}{2^{2n-1} n! (n-1)!}\\
        & = \frac{(-1)^{n-1}}{2^{2n-1}n}\binom{2n-2}{n-1}
    \end{align}
    On en déduit 
    \begin{align}
        C(x) &= -\frac{1}{2x}\sum_{n=1}^\infty \frac{(-1)^{n-1}}{2^{2n-1}n}\binom{2n-2}{n-1}(-4x)^n\\
        & =\sum_{n=1}^\infty \frac{(-1)^{n}}{2^{2n}n}\binom{2n-2}{n-1}(-1)^n2^{2n}x^{n-1}\\
        & = \sum_{n=1}^\infty \frac{1}{n}\binom{2n-2}{n-1}x^{n-1}\\
        & = \sum_{n=0}^\infty \frac{1}{n+1}\binom{2n}{n}x^n
    \end{align}
    Par unicité de la série entière, on a \(c_n = \frac{1}{n+1}\binom{2n}{n}\).
\end{proof}
\begin{corollary}
    \begin{equation}
        c_n \sim \frac{4^n}{n^{3/2}\sqrt{\pi}}
    \end{equation}
\end{corollary}
\begin{proof}
    On rappelle la formule de Stirling :
    \begin{equation}
        n! \sim \sqrt{2\pi n}\left(\frac{n}{e}\right)^n
    \end{equation}
    On a alors 
    \begin{align}
        c_n & = \frac{1}{n+1}\binom{2n}{n}\\
        & = \frac{1}{n+1}\frac{(2n)!}{n!n!}\\
        & \sim \frac{1}{n+1}\frac{\sqrt{4\pi n}(2n)^{2n}e^{-2n}}{2\pi n n e^{-2n}}\\
        & \sim \frac{4^n}{n^{3/2}\sqrt{\pi}}.
    \end{align}
\end{proof}
\end{document}