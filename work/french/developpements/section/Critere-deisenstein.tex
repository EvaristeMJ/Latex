\documentclass[../main.tex]{subfiles}
\begin{document}
\section{Critère d'Eisenstein}
\begin{abstract}
    Le critère d'Eisenstein est un critère d'irréductibilité d'un polynôme. Dans sa forme
     la plus connue, il s'énonce ainsi : soit \(P = \sum_{i=0}^d a_i X^i\) un polynôme à coefficients dans \(\Z\).
     S'il existe un nombre premier \(p\) tel que \(p\) divise tous les coefficients sauf le dernier et que 
     \(p^2\) ne divise pas le premier coefficient, alors \(P\) est irréductible dans \(\Q[X]\).

     Nous en donnerons une généralisation dans le cas des polynômes à coefficients dans un anneau intègre.
\end{abstract}
\begin{definition} Soit \(\mathcal{A}\) un anneau commutatif. Un idéal \(I\) est dit premier
    si pour tous \(a,b\in \mathcal{A}\), si \(ab\in I\), alors \(a\in I\) ou \(b\in I\).
    \begin{remark} De manière équivalente, un idéal \(I\) est dit premier si \(\mathcal{A}/I\) définit par
        passage au quotient un anneau intègre.
    \end{remark}
\end{definition}
\begin{theorem}[Critère d'Eisenstein] Soit \(\mathcal{A}\) un anneau intègre. Soit \(P = \sum_{i=0}^d a_ix^i\in \mathcal{A}[x]\).
    S'il existe un idéal premier \(I\) de \(\mathcal{A}\) tel que 
    \begin{itemize}
        \item \(\forall i\in \{0,...,d-1\}, a_i\in I\),
        \item \(a_d\notin I\),
        \item \(a_0\notin I^2\) (c'est-à-dire \(a_0\) n'est pas le carré d'un élément de \(I\)),
    \end{itemize}
    Alors \(P\) est irréductible dans \(\mathcal{A}[x]\).
\end{theorem}
\begin{proof} Supposons que \(P = RQ\) avec \(\deg(R) = m\) et \(\deg(Q) = d-m\) avec \(\deg(R) \ge 1\).
    Par réduction modulo \(I\) (i.e. considérer \(\mathcal{A}/I[x]\)), on a 
    \begin{equation}
        P = RQ = a_d x^d\; \mathrm{mod}\;I.
    \end{equation}
    Or \(\mathcal{A}/I\) est intègre donc \(\mathcal{A}/I[x]\) est intègre. On en déduit que les réductions de \(R\) et \(Q\)
    sont de la forme \(R = bx^{m} \; \mathrm{mod}\;I\) et \(Q = cx^{d-m}\; \mathrm{mod}\;I\). En particulier, on en déduit que
    \(r_0\) et \(g_0\) sont dans \(I\). Puisque \(a_0 = r_0g_0\), on en déduit que \(a_0\in I^2\). Cela contredit l'hypothèse du théorème,
    on en déduit que \(P\) est irréductible.
\end{proof}
Le théorème d'Eisenstein pour les polynômes à coefficients dans \(\Z\) est un peu plus fort que le théorème général puisqu'on obtient une irréductibilité dans \(\Q[X]\) et non dans \(\Z[X]\).
C'est l'objet de la proposition suivante, qui montre que les deux théorèmes sont équivalents.
\begin{proposition} Un polynôme \(P\in \Z[X]\) est irréductible dans \(Q[X]\) si et seulement si il est
    irréductible dans \(\Z[X]\).
\end{proposition}
\begin{proof} Bien sûr, si \(P\) est irréductible dans \(\Q[X]\), il est aussi dans \(\Z[X]\). Supposons désormais
    qu'il le soit dans \(\Z[X]\). Supposons qu'il existe \(R,Q\in \Q[X]\) tels que \(P = RQ\).

    Il existe \(q,r\in\Z\) tel que \(qQ\in \Z[X]\) et \(rR\in \Z[X]\). On peut ensuite écrire
    \begin{equation}
        qrP = qR rQ = c(qR)R'c(rQ)Q'
    \end{equation}
    où \(R',Q'\) sont des polynômes à coefficients entiers et \(c(R),c(Q)\) sont les pgcd des coefficients de \(R\) et \(Q\).
    On en déduit que 
    \begin{equation}
        qrc(P) = c(qR)c(rQ)
    \end{equation}
    d'où \(qrP = qrc(P)R'Q'\) et finalement que \(P = c(P)R'Q'\). En conclusion, \(P\) est irréductible dans \(\Z[X]\) puisque \(R',Q'\) sont de degrés supérieurs à \(1\),
    et \(c(P)\) n'est qu'une constante entière.
\end{proof}
\begin{corollary}[Critère d'Eisenstein] Soit \(P = \sum_{i=0}^d a_i X^i\) un polynôme à coefficients dans \(\Z\).
    S'il existe un nombre premier \(p\) tel que \(p\) divise tous les coefficients sauf le dernier et que 
    \(p^2\) ne divise pas le premier coefficient, alors \(P\) est irréductible dans \(\Q[X]\).
\end{corollary}
\begin{proof} On applique le théorème d'Eisenstein avec \(\mathcal{A} = \Z\) et \(I = (p)\).
    Ainsi, \(P\) est irréductible dans \(\Z[X]\) donc dans \(\Q[X]\) par la proposition précédente.
\end{proof}
Le critère d'Eisenstein est un critère d'irréductibilité puissant, en particulier lorsque l'on 
considère les fermés de Zariski. L'idée est de considérer \(\mathcal{A}[x,y]\) comme \(\mathcal{A}[x][y]\) et d'appliquer le critère
d'Eisenstein sur \(\mathcal{A}[x]\).
\begin{example} Soit \(f = y^2 + yx^2 + x\). On peut considérer \(I = (x)\). \(f_0 = x \in I\) et \(f_0 = x \not\in I^2\), 
    \(f_1 = x^2 \in I\) et \(f_2 = 1 \not\in I\). On en déduit que \(f\) est irréductible dans \(\C[x][y]\).
\end{example}
De manière générale, on a le corollaire suivant :
\begin{corollary} Soit \(f\in \mathcal{A}[x,y]\) sous la forme \(f = \sum_{i=0}^d f_i(x)y^i\). 
    Si les \(f_i\) sont premiers entre eux et qu'il existe un polynôme irréductible \(p(x)\) tel que 
    \(p(x)\) divise tous les \(f_i\) sauf le dernier et que \(p^2(x)\) ne divise pas \(f_0\), alors \(f\) est irréductible dans \(\mathcal{A}[x,y]\).
\end{corollary}
Le critère d'Eisenstein s'applique plus souvent sur des polynômes à coefficients entiers.
On peut par exemple établir l'irréductibilité du \(p\)-ième polynôme cyclotomique.
\begin{equation}
    \Phi_p(X) = \frac{X^p-1}{X-1} = X^{p-1} + X^{p-2} + \ldots + 1.
\end{equation}
\begin{corollary}
    Le \(p\)-ième polynôme cyclotomique est irréductible dans \(\Q[X]\).
\end{corollary}
\begin{proof} On calcule d'abord \(\Phi_p(X+1)\)
    \begin{equation}
        \Phi_p(X+1) = \frac{(X+1)^P-1}{X} = \sum_{i=1}^p \binom{p}{i}X^{i-1}.
    \end{equation}
    D'après le critère d'Eisenstein avec \(p\), \(\Phi_p(X+1)\) est irréductible dans \(\Q[X]\).
    (\(p\) divise \(\binom{p}{i}\) pour \(i\in\{1,...,p-1\}\), \(p^2\) ne divise pas \(\binom{p}{1}\) et \(p\) ne divise pas \(\binom{p}{p}\)). 
    Si \(\Phi_p\) était réductible, alors il existerait \(R,Q\in \Q[X]\) tels que \(\Phi_p = RQ\). On aurait alors
    \begin{equation}
        \Phi_p(X+1) = R(X+1)Q(X+1) = R'(X)Q'(X).
    \end{equation}
    Ce qui contredirait l'irréductibilité de \(\Phi_p(X+1)\).
\end{proof}
On conclut par un dernier corollaire intéressant.
\begin{corollary} \(\Q[X]\) admet des polynômes irréductibles de degré arbitrairement grand.
\end{corollary}
\begin{proof} On pose \(P_n = X^n - 2\). On a directement par le critère d'Eisenstein que \(P_n\) est irréductible dans \(\Q[X]\).
\end{proof}

\end{document}