\documentclass[../main.tex]{subfiles}
\begin{document}
\section{Formule de Mobius et dénombrements des polynômes irréductibles}
\begin{definition} Soit \(n\in\N\), on définit la fonction de Möbius \(\mu:\N\to\Z\) par
    \begin{equation}
        \mu(n) = \begin{cases}
            0 & \text{si } n \text{ n'est pas produit de carrés,}\\
            (-1)^k & \text{si } n \text{ est produit de } k \text{ nombres premiers distincts.}
        \end{cases}
    \end{equation}
\end{definition}
\begin{proposition} Soit \(n\ge 2\), alors
    \begin{equation}
        \sum_{d|n} \mu(d) = 0.
    \end{equation}
\end{proposition}
\begin{proof} On décompose \(n\) en produit de facteurs premiers 
    \begin{equation}
        n = \prod_{i=1}^k p_i^{\alpha_i}.
    \end{equation}
    On peut ensuite décomposer la somme, en tenant compte que \(\mu(d) = 0\) si deux diviseurs premiers divisent \(d\),
    \begin{align}
        \sum_{d|n} &= \mu(1) + \sum_{i=1}^k \mu(p_i) + \sum_{i\neq j} \mu(p_ip_j) + \ldots + \mu(p_1\ldots p_n)\\
        &= \sum_{k=0}^n (-1)^k \binom{n}{k}\\ 
        &= 0.
    \end{align}
\end{proof}
\begin{theorem}[Formule d'inversion de Möbius] Soit \(f\) une fonction arithmétique, on pose 
    \(g(n) = \sum_{d|n} f(d)\). Alors, on a
    \begin{equation}
        f(n) = \sum_{d|n}\mu\left(\frac{n}{d}\right) g(d) = \sum_{d|n}\mu(d)g\left(\frac{n}{d}\right).
    \end{equation}
\end{theorem}
\begin{proof} On a
    \begin{align}
        \sum_{d|n}\mu(d)g\left(\frac{n}{d}\right) & = \sum_{d|n}\mu(d)\sum_{e|\frac{n}{d}}f(e)\\
        & = \sum_{de|n}f(e)\mu(d)\\
        & = \sum_{e|n}f(e)\sum_{d|\frac{n}{e}}\mu(d)\\
        & = f(n).
    \end{align}
    car \(\sum_{d|\frac{n}{e}}\mu(d)\) vaut \(0\) si \(e\neq n\) et \(1\) sinon.
\end{proof}
\subsection*{Application aux dénombrements de polynômes irréductibles sur \(\F_q\)}
\begin{definition} Soit \(\K\) un corps, un polynôme \(f\) de \(\K[x]\) est dit irréductible si
    il n'est pas le produit de deux polynômes non inversibles de \(\K[x]\), i.e.
\begin{equation}
    f = gh \implies g\in  \K[x]^\times\text{ ou } h\in \K[x]^\times.
\end{equation}
\end{definition}
\begin{proposition}\label{prodirr} Soit \(q\) une puissance d'un nombre premier. On note \(I(n,q)\) 
    l'ensemble des polynômes irréductibles unitaires de degré \(n\) sur \(\F_q\). Alors, on a 
    \begin{equation}
        x^{q^n} - x = \prod_{d|n}\left(\prod_{f\in I(d,q)}f\right)
    \end{equation}
\end{proposition}
\begin{proof}
    Soit \(d|n\), on considère \(g \in I(d,q)\). On considère ensuite \(K\) le corps
    de rupture de \(g\) sur \(\F_q\). \(K\) est une extension de degré \(d\) qui sera isomorphe
    à \(\F_{q^d}\), en considérant \(K\) comme \(\F_q-\)ev de dimension \(d\) (en prenant \(\alpha\) comme élément primitif).
    On sait que \(\alpha^{q^d} = \alpha\) donc \(\alpha\) est une racine de \(x^{q^n}-x\). 
    On en déduit donc que \(f\) divise \(x^{q^n} - x\).

    Réciproquement, supposons que \(f\) est un facteur irréductible de \(x^{q^n} - x\).
    On note \(d\) le degré de \(f\). Par ailleurs, \(f\) est scindé dans \(\F_{q^n}\) donc les racines de \(f\) sont dans \(\F_{q^n}\).
    En particulier, si on considère le corps de rupture \(K\) de \(f\) sur \(F_q\), on sait que \(K\) est une extension de \(\F_q\) de degré \(d\).
    \begin{equation}
        \F_q \subset K\cong \F_{q^d} \subset \F_{q^n}.
    \end{equation}
    On peut aussi construire \(\F_{q^n}\) comme un corps de rupture pour un polynôme irréductible de degré \(k\), d'où
    \begin{equation}
        [\F_{q^n}:\F_q] = [\F_{q^n}:\F_{q^d}][\F_{q^d}:\F_q].
    \end{equation}
    D'où \(d|n\).
\end{proof}
\begin{theorem}
    \begin{equation}
        |I(n,q)| = \frac{1}{n}\sum_{d|n}\mu\left(\frac{n}{d}\right)q^{d}.
    \end{equation}
\end{theorem}
\begin{proof} Utilisons la Proposition \ref*{prodirr} et comparons les degrés de chaque côté :
    \begin{equation}
        q^n = \sum_{d|n}|I(d,q)|d.
    \end{equation}
    Appliquons ensuite la formule d'inversion avec \(n\mapsto |I(n,q)|\). On en déduit que
    \begin{equation}
        n|I(n,q)| = \sum_{d|n}\mu\left(\frac{n}{d}\right)q^{d}.
    \end{equation}
    Il suffit ensuite de diviser par \(n\) des deux côtés pour obtenir le résultat attendu
    \begin{equation}
        |I(n,q)| = \frac{1}{n}\sum_{d|n}\mu\left(\frac{n}{d}\right)q^{d}.
    \end{equation}
\end{proof}
\begin{remark} On récupère aussi un équivalent de \(|I(n,q)|\).
    \begin{equation}
        |I(n,q)| \sim \frac{q^n}{n}.
    \end{equation}
    Pour cela, il suffit de remarquer les inégalités suivantes
    \begin{align}
        \lvert\frac{1}{n}\sum_{d|n, d<n}\mu\left(\frac{n}{d}\right)q^{d}\rvert & \le \frac{1}{n}\sum_{d|n,d<n}\lvert\mu\left(\frac{n}{d}\right)q^{d}\rvert\\
        & \le \frac{1}{n}\sum_{d|n,d<n}q^{d}\\
        & \le \frac{1}{n}\sum_{d=1}q^{\lfloor \frac{n}{2} q^d\rfloor}\\
        & = \frac{1}{n} \frac{q^{\lfloor \frac{n}{2} q^d\rfloor+1}-1}{q-1}
    \end{align}
    qui est un \(o(q^n)\). Ainsi, \(|I(n,q)| = \frac{q^n}{n}+o(q^n)\), d'où \(|I(n,q)| \sim \frac{q^n}{n}\).
\end{remark}
\end{document}