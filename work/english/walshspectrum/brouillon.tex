\documentclass[12pt]{article}
\usepackage{amsmath, amssymb, amsthm}
\usepackage{geometry}
\geometry{a4paper, margin=1in}
\usepackage{hyperref}
\usepackage{graphicx}
\usepackage{enumitem}

\theoremstyle{definition}

\newtheorem{theorem}{Theorem}[section]
\newtheorem{definition}[theorem]{Definition}
\newtheorem{corollary}[theorem]{Corollary}
\newtheorem{lemma}[theorem]{Lemma}
\newtheorem{proposition}[theorem]{Proposition}
\newtheorem{example}[theorem]{Example}
\newtheorem{remark}{Remark}

\newcommand{\C}{\mathbb{C}}
\newcommand{\R}{\mathbb{R}}
\newcommand{\Z}{\mathbb{Z}}
\newcommand{\N}{\mathbb{N}}
\newcommand{\Q}{\mathbb{Q}}
\newcommand{\F}{\mathbb{F}}
\newcommand{\W}{\mathsf{W}}

\title{On the Cardinal of the Support of Walsh for Functions of few Variables}
\author{Maxence Jauberty}
\begin{document}
\maketitle
Boolean functions play a crucial role in cryptography and error-correcting codes due to their
diverse applications and rich mathematical properties. One such property, the Walsh 
transform, is a Fourier-Hadamar transform that provides valuable insights into the 
spectral behavior of Boolean functions. The Walsh support of a Boolean functions, defined 
as the set of points where the Walsh transform is nonzero, offers further structural information.
Despite its signifiance, the Walsh support remains relatively underexplored.
\section{Definitions}
\begin{definition} Let \(f : \F^n_2 \to \F_2\) be a Boolean function and \(a\in \F_2^n\), 
    the Walsh transform in \(a\) is defined as :
    \begin{equation*}
        \W_f(a) := \sum_{x\in\F_2^n}(-1)^{f(x)+a\cdot x},
    \end{equation*}
    and the Walsh support is:
    \begin{equation*}
        \W_{\mathrm{supp}}(f) := \{a\in \F_2^n,\;\W_f(a)\neq 0\}.
    \end{equation*}
\end{definition}
\end{document}