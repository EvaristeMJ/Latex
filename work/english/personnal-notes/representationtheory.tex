\documentclass[12pt]{article}
\usepackage{amsmath, amssymb, amsthm}
\usepackage{geometry}
\geometry{a4paper, margin=1in}
\usepackage{hyperref}
\usepackage{graphicx}
\usepackage{enumitem}

\theoremstyle{definition}

\newtheorem{theorem}{Theorem}[section]
\newtheorem{definition}[theorem]{Definition}
\newtheorem{corollary}[theorem]{Corollary}
\newtheorem{lemma}[theorem]{Lemma}
\newtheorem{proposition}[theorem]{Proposition}
\newtheorem{example}[theorem]{Example}

\newcommand{\C}{\mathbb{C}}
\newcommand{\R}{\mathbb{R}}
\newcommand{\Z}{\mathbb{Z}}
\newcommand{\N}{\mathbb{N}}
\newcommand{\Q}{\mathbb{Q}}

\title{Representation Theory}
\author{Maxence Jauberty}
\date{\today}

\begin{document}
\maketitle
\tableofcontents
\begin{abstract}
    This document gathers my personnal notes on representation theory of finite groups. 
    It is mostly based on two references : The Symmetric Group by Bruce Sagan and Harmonic Analysis on Finite Groups by Ceccherini-Silberstein, Scarabotti and Tolli.
\end{abstract}
\section{Basic Definitions and Motivations}
The notion of representation arises quite frequently in mathematics. Morally, a representation is 
a way of saying that whatever the abstract object we are studying really is, it can be thought of 
as a concrete object of that said stucture. In other words, our abstract definition probably
catch the essence of the object we are studying. Representation theorems are usually great theorems.
For example, Cayley's theorem says that every group can be thought as a subgroup of the permutation group.

Here is an unformal definition of what we call a representation in mathematics. However, 
representation theory is not that general and focus mainly on the structure of group actions.
\begin{definition}
    Let \(G\) be a group and \(X\) be a set. We say that \(G\) acts on \(X\) if there is a map
    \[
        G \times X \ni (g,x) \mapsto g \cdot x\in X
    \]
    such that 
    \begin{enumerate}
        \item \(e \cdot x = x\) for all \(x \in X\),
        \item \(g \cdot (h \cdot x) = (gh) \cdot x\) for all \(g,h \in G\) and \(x \in X\).
    \end{enumerate}
\end{definition}
\begin{example}
    \begin{enumerate}
        \item The group \(\mathfrak{S}_n\) acts on the set \(\{x_1,\dots,x_n\}\) with the action defined by 
        \begin{equation*}
            \sigma \cdot x_i = x_{\sigma(i)}
        \end{equation*}
        \item The group \(\mathcal{GL}_n(\C)\) acts on a vector space \(V\) with the action defined by 
        \begin{equation*}
            \mathcal{GL}_n(\C)\times V\ni (A,v)\mapsto Av\in V
        \end{equation*}
    \end{enumerate}
\end{example}
The second example should bring a whole new intuition to the notion of group actions as a 
generalization of the notion of linear transformations. In particular, it is easier to see that 
\(X\) somehow inherits a structure from \(G\). Representation theory is all about studying 
the action structure through concrete objects from linear algebra.
\begin{definition}
    A representation of \(G\) over a vector space \(V\) is an action of \(G\) on \(V\) defined by 
    \begin{equation}
        G \times V \ni (g,v) \mapsto \rho(g)v \in V
    \end{equation}
    where \(G\ni g \mapsto \rho(g)\in \mathcal{GL}_d\) is a group homomorphism.
\end{definition}
\section{Gelfand Pair}

\end{document}